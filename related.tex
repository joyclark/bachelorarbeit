\section{Related Work}

The focus of this work was to apply existing graph algorithms to the problem of visualizing a state space. This is the approach currently taken by the ProB software, although we are proposing changing the graph engine from GraphViz\footnotemark[1] to a JavaScript powered visualization. CSP-M\cite{Fontaine11} is another tool which generates DOT file to visualize the state space associated with CSP specifications.

Another approach to visualizing state spaces is the application of different algorithms to the state space in order to simplify it without losing key characteristics of the state space. Several of these algorithms have already been integrated into ProB \cite{LeTu05_8}. In the course of this work, we have integrated support for two of the algorithms present in ProB. However, there are other algorithms available in ProB, such as the DFA-Abstraction algorithm presented in \cite{LeTu05_8}, which have not been supported in this work. 

There are also several tools developed for the specific purpose of visualizing state spaces. Van Ham et al. \cite{Ham02} have researched the problem of visualizing labeled transition systems and developed the LTSView\footnote{http://www.mcrl2.org/release/user\_manual/tools/ltsview.html} tool for visualizing the structure of state spaces. This tool produces a 3D representation of a state space that the user can inspect. In order to simplify the visualizations that are created by LTSView, work has been done to develop a method of converting 3D models of labeled transition systems to 2D \cite{Pretorius2005}. The StateVis tool is the result of this research\footnote{http://www.win.tue.nl/vis1/home/apretori/statevis/}. These tools can help users to get a rough idea of the feel of the whole state space. However, they do not allow for a close inspection of interesting parts of the state space. In order to provide a solution to this problem, Pretorius and van Wijk have proposed a method for interacting with the visualization of state transition graphs \cite{Pretorius2006}. The tools NoodleView\footnote{http://www.comp.leeds.ac.uk/scsajp/applications/noodleview/} and its successor DiaGraphica\footnote{http://www.comp.leeds.ac.uk/scsajp/applications/diagraphica/} attempt to apply this method.
