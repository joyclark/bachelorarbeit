\section{Related Work}

The focus of this work was to apply existing graph algorithms to the problem of visualizing a state space. However, a great deal of work has been done in developing algorithms to simplify the state space so that it can be better visualized. Two algorithms have already been integrated into ProB \cite{LeTu05_8}. One of these, the signature merge algorithm, has already been presented in this work. However, the DFA-Abstraction algorithm presented there has not yet been integrated into the new tool.

Several tools exist specifically for the visualization of directed graphs. Walrus\footnote{http://www.caida.org/tools/visualization/walrus/} and GraphViz\footnote{http://www.graphViz.org}, which has already been introduced in this paper, are two tools that can be used for graph visualization. Some tools which require the visualization of labeled transition systems use the graph visualization features that are provided by these tools. As we have seen, ProB is one of these tools. CSPM\footnote{http://hackage.haskell.org/package/CSPM-cspm} is another tool which supports the generation of DOT files in order to visualize the state space associated with CSP specifications.

However, there are also several tools developed for the specific purpose of visualizing state spaces. Van Ham et al. \cite{Ham02} have considered the problem of visualizing labeled transition systems and developed the LTSView\footnote{http://www.mcrl2.org/release/user\_manual/tools/ltsview.html} tool for visualizing the structure of state spaces. This tool produces a 3D representation of the state spaces. With the LTSView tool in view, work has been done to develop a method of converting 3D models of labeled transition systems to 2D \cite{Pretorius2005}. The StateVis tool is the result of this research\footnote{http://www.win.tue.nl/vis1/home/apretori/statevis/}.
