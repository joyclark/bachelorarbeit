\section{Future Work}

One of the main features we implemented in this work was the creation of a visualization framework. In order to create new visualizations, some work by a developer will have to be done, but a lot of the elements of the existing visualizations can be adapted to minimize the work that the developer will have to do. As we have mentioned before, we have made a JavaScript library avaliable to make it easier to create new D3 powered visualizations within ProB. A developer will also be able to extend the session based servlet that we have created to create the visualization engine that he needs for his new visualization.

ProB 2.0 still does not have all of the visualization elements that are available in the Tcl/Tk version of ProB. Therefore, future work will entail the use of this existing visualization framework in order to recreate these visualization in ProB 2.0. For instance, one possible visualization could be the implementation of a graphical representation of a state within ProB 2.0. Another could be the implementation of a visualization of the refinement hierarchy that is present in Classical B and Event-B models.

One of the projects in ProB 2.0 that is currently underway is the development of a worksheet element. This will serve as documentation and a means to run groovy scripts within the application. In the future, it should be possible to embed the visualizations created in the scope of this work within the worksheet element. 

The state space visualization and the visualization of a formula over time are not language specific. It is possible to use the visualizations for any specification language that ProB supports. For the visualization of the breakdown of a formula, however, this is not the case. The current implementation supports only Classical B and Event-B formulas. In the future, work could be done to support formulas from other formalisms.

Current visualizations will also need to be maintained and updated to add functionality. For instance, it might be possible to integrate the proposed state visualization with the existing state space visualization. Then, when a user would select a state, a window would pop up displaying the state visualization for that particular state.