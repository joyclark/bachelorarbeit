\section{Conclusion}

During the course of this work, we were able to fulfill all of our objectives. We were able to create three new data visualizations that will be useful for users of ProB. These visualizations are generated dynamically from data that is produced by ProB during consistency checking and the animation of formal specifications. They are updated automatically whenever any changes in the state space or in the current animation take place. We have also created a way for the user to customize the appearance of the visualizations.

The visualization framework that we have created is extensible. It should be possible to create new data visualization that will fit into the existing framework. It should also be possible to adapt the existing visualizations to add functionality. For instance, the decision to support GraphViz visualizations in the state space visualization took place relatively late in the development process. Using the existing visualization framework, it was possible to implement this feature in a relatively short period of time. This shows that the visualization framework is relatively flexible and can be easily extended.

We have shown the feasibility of using D3 to create data visualizations within ProB. By using the browser widgets available in the Eclipse RCP application, it was simple to integrate the JavaScript visualizations into ProB 2.0. These views appear to be native to the Eclipse platform, but there is a much higher level of user interaction possible with web applications than there is with native Eclipse views. For example, the zooming and panning capability available in all of the visualizations is a pure JavaScript function that is made available from D3. 

All in all, adding data generated visualizations into the ProB 2.0 application has made the program more useful for the users. It is now not only possible for the user to extract and analyze textual data from ProB, but also to view and interact with a visualization based on the data.


