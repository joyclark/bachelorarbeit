\section{Conclusion}

Over the course of this work, we have shown the feasibility of using D3 to create data visualizations within ProB. It is not only possible to create the desired visualizations, it is also easy to adapt the servlets so that the visualizations are updated to reflect changes made in the course of model animation. The styling for these visualizations can be easily defined by the user. This gives the user a large amount of control over the visualization.

The visualization of the state space was the focus of this work. The visualization that was created is interactive and is automatically updated as soon as states are calculated and cached in the state space. It also can handle relatively large state spaces (TEST THIS OUT TO DETERMINE HOW LARGE!!!!). 

The decision to support GraphViz visualizations in the state space visualization in addition to D3 visualizations took place relatively late in the development process. Using the existing visualization framework, it was possible to implement this feature in a relatively short period of time. This shows that the visualization framework is relatively flexible and can be easily extended.