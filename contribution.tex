\section{Contribution}

\subsection{Visualization framework}

\subsubsection{Structure}

One of the main issues that we had to be deal with at the beginning of the development process was the issue of how to integrate the visualizations into ProB 2.0. A functioning jetty server was already available, so we had to integrate our JavaScript visualizations into the existing framework. We ran into problems at the beginning, because a Java Servlet\footnote{http://docs.oracle.com/javaee/6/api/javax/servlet/Servlet.html} is a singleton object. Ideally, we would have a unique servlet for every visualization because each visualization has its own set of data that needs to be visualized. In order to solve this problem, we created a session based servlet. When the user wants to create a new visualization, a unique session id is created. This is sent to the session based servlet, which creates a visualization engine responsible for the data calculation. When the visualization is initialized, it sets up a polling interval to ask the servlet for either the full dataset or for the changes since the last poll (see Figure \ref{communication}). When the visualization polls the servlet, it includes its unique session id. The servlet then forwards the request to the visualization engine responsible for the data calculations which then returns the correct dataset.

The servlet is also connected to ProB 2.0 through the listener framework. Depending on the data that is needed for a particular visualization, the servlet registers either to receive notifications about changes in the current animation or about changes in the state space (see Figure \ref{programFlow}). Every time the servlet receives a new notification, the data needed for the visualization is recalculated.

\begin{center}
\begin{figure}[h!]
\centering
\includegraphics[width=14cm]{bilder/communication.pdf}
\caption{Visualizations poll the servlet to receive the dataset that needs to be visualized.}
\label{communication}
\end{figure}
\end{center}

\begin{center}
\begin{figure}[h!]
\centering
\includegraphics[width=14cm]{bilder/programFlow.pdf}
\caption{Model of the program flow.}
\label{programFlow}
\end{figure}
\end{center}

The visualizations also support some persistence. For each visualization engine that is created, the settings for the visualization are saved with the given session id. If, over the course of the user's interaction with the visualization, the settings for the visualization change, the cached settings are updated. For example, if the user is visualizing the state space and has chosen to create a transition diagram for a given expression, both the expression and the type of graph that the user is viewing are saved. When the Eclipse application closes, the URL for each visualization that is open is saved (e.g. \texttt{http://localhost:8080/statespace\_servlet/?init=\emph{sessionId}}). The next time that the application is started, the cached settings for the given session id will be retrieved, and the visualization will be restored from the settings.

\subsubsection{User customization of the visualizations}

We wanted the user to be able to customize the appearance of the visualizations. One of the main advantages of the D3 visualization framework is the flexibility that it provides for the developer. Using D3 selectors, it is possible for a developer to select and change the attributes of any of the elements of the visualization. We wanted the user to also be able to manipulate the visualization from within ProB. 

In order to allow the user to directly manipulate the DOM of the visualization and thereby customize the appearance of the visualization, we decided to lift the functionality of the D3 selectors from the JavaScript level into the Java application. For this purpose, we defined a \texttt{Transformer} object. The user can create a \texttt{Transform} by defining a selector string based on the W3C Selectors API\footnote{http://www.w3.org/TR/selectors-api/} and by calling the \texttt{set} method. The \texttt{set} method saves the name of the attribute that should be changed and the value to which the specified attribute should be changed. Because the \texttt{set} method returns the instance of the \texttt{Transformer}, it is possible chain the method calls together (see Lisiting \ref{javaTransformer}).

\lstset{language=java}
\begin{lstlisting}[caption=Define a \texttt{Transformer} in Java,label=javaTransformer]
// Select elements with ids "sroot" and "s1" and set their fill and stroke attributes
Transformer t = new Transformer("#sroot,#s1").set("fill","red").set("stroke","gray");
\end{lstlisting}

Every visualization engine contains a list of \texttt{Transformers} that is sent to the visualization every time that it is polled. The visualization can then read the selector and the attributes saved in the \texttt{Transformer} and apply the desired actions to the DOM. We now needed a way for the user to create \texttt{Transformers} and add them to the visualization. It is possible to add a \texttt{Transformer} to a visualization by calling the \texttt{apply} method in the visualization engine responsible for the visualization. When a new visualization engine is created, a variable with a reference to the object is loaded into the Groovy execution engine so that the user can access the visualization engine from within the Groovy REPL. We have also made the method \texttt{transform} available in the Groovy REPL so that the user can easily create \texttt{Transformers} (see Listing \ref{transformer}).

\lstset{language=java}
\begin{lstlisting}[caption=Create and apply a \texttt{Transformer} in the Groovy REPL,label=transformer]
// Select elements with ids "sroot" and "s1" and set their fill and stroke attributes
x = transform("#sroot,#s1") {
        set "fill", "red"
        set "stroke", "gray"
}

// Apply to visualization
viz0.apply(x)
\end{lstlisting}

It is also possible to harness the power of the Groovy closure in order to create a Transformer that can be parameterized (see Listing \ref{transWclosure}).

\begin{lstlisting}[caption=Use Groovy closures to generate \texttt{Transformers},label=transWclosure]
// Create a closure that can be parameterized
colorize = { selection, color ->
                transform(selection) {
                    set "fill", color
                }    
           }

// Color elements "sroot" and "s1" green
viz0.apply(colorize("#sroot,#s1", "green"))
\end{lstlisting}

The downside to this mechanism is that the user needs to have a good idea about the internal representation of the DOM in order to manipulate the visualizations. This mechanism is especially interesting for the state space visualization, so we spent more time defining the ids for the DOM elements. The SVG objects that are created using D3 are generated using the state ids and transition ids that are assigned by ProB. The mechanism for the definition of these ids can be seen in Table \ref{tab:elements}.

\begin{table}[h!]
\caption{Generated element ids for elements in the state space visualization.}
\label{tab:elements}
\begin{center}
\begin{tabular}{ | l | l | }
\hline
\textbf{Element} & \textbf{Generated Element Id} \\ \hline \hline
transition & \texttt{t + }\emph{transition id} \\ \hline
text on transition & \texttt{tt + }\emph{transition id} \\ \hline
state & \texttt{s + }\emph{state id} \\ \hline
text on state & \texttt{st + }\emph{state id} \\ \hline
\end{tabular}
\end{center}
\end{table}

It is also possible in ProB to filter a state space based on a predicate. This means that it is possible to give ProB a predicate and receive a list of state ids for which the given predicate holds. Ideally, however, we would want a \texttt{Transformer} based on a given predicate to be updated in the case that the state space changes.. For this purpose, we have introduced a \texttt{Transformer} that will be updated in the case that new states are added to the state space. The user can create such a \texttt{Transformer} in the Groovy REPL with the \texttt{transform} method by specifying a predicate and a state space object (see Listing \ref{dynamicTransformer}). Currently, this will only update the state objects in the DOM, but in the future, we will be able to apply the same concept for all the different elements in the DOM.

\begin{lstlisting}[caption=Create a \texttt{Transformer} based on the states that match a given predicate,label=dynamicTransformer]
// Define your formula
predicate = "active\\/waiting={}" as ClassicalB

// Create a Transformer that will filter the given state space (saved in space0) according to the given predicate
x = transform(predicate, space_0 ) {
		set "fill", "blue"
		set "stroke", "white"
}

// Apply the Transformer to the visualization
viz0.apply(x)
\end{lstlisting}

\subsubsection{Extensibility}

In addition to creating visualizations that are useful to the user, we also wanted to make it easy for other developers to create similar visualizations. In order to help with this, we encapsuled certain elements common to all of the visualizations into a separate script. In order to have access to these elements, a developer simply needs to include the script before that of his visualization. Currently, there are two main elements included in this script. Firstly, the user can use the \texttt{createCanvas} function to create a D3 selection that includes support for zooming (see Listing \ref{canvas}). Secondly, if the user wants to be able to apply the \texttt{Transformers} that are described in the last section, there is a built in function to apply a list of \texttt{Transformers} to the visualization (see Listing \ref{applyStyling}). In the future, it will also be possible to add more functions to this script to make it even easier to create visualizations for ProB.

\begin{lstlisting}[caption=Append a D3 selection to an element that includes support for zooming,label=canvas]
var width = 600, height = 400;
var svg = createCanvas("#elementId", width, height);

// Append all further elements for the visualization to this selection
svg.append(...);
\end{lstlisting}

\begin{lstlisting}[caption=Apply list of \texttt{Transformers} received from servlet,label=applyStyling]
// The servlet has been polled, and a response has been received
var styling = response.styling;

// ... Render the visualization
// ... Then apply the user defined styling
applyStyling(styling);
\end{lstlisting}

\subsection{Visualization of the State Space}

\subsubsection{Main visualization}

When a state space visualization is opened, the visualization engine responsible for the state space visualization takes the state space associated with the current animation and extracts the information about the states and transitions contained within the graph. This information is then processed using the D3 force layout and rendered to create a visualization (see Figure \ref{zoomedOut}). As a basis for this visualization, we used the Force Directed Graph example created by Michael Bostock (see Appendix \ref{appendix:force}). A user can open the state space visualization by selecting \textsf{Analyze > Open State Space Visualization} in the \textsf{ProB} menu of the ProB 2.0 application. This will open a new visualization of the state space for the current animation. 

Unfortunately, in some cases the state space contains an extremely large amount of information that has to be processed. This includes the values of the variables and the invariant for every state in the graph and the names and parameters of the transitions that correspond to every edge in the graph. Because of this, it is rather difficult to create a useful visualization of the whole state space because the user not only wants to inspect the structure of the state space as a whole but also the individual states. 

\begin{center}
\begin{figure}[h!]
\centering
\includegraphics[width=14cm]{bilder/ss.png}
\caption{Visualization of a partially explored state space for the Scheduler example}
\label{zoomedOut}
\end{figure}
\end{center}

As a proposed solution of this problem, we used the zoom functionality that is available in D3. The main problem was that if the visualization of the nodes was large enough for the user to read the values of the variable at the given state, it would no longer be possible to see the state space as a whole. Instead of trying to meet both requirements at once, we simply made the text that is printed on the state and transition objects very small. When the visualization is created, the user can view the graph to see how the state space appears as a whole. The text for the individual states, however, is virtually indiscernable. If the user wants to inspect a particular state, they can do so by zooming into the visualization. The text is then larger, and the user can see the values of the variables for the given state and the outgoing transitions (see Figure \ref{onestate}). Then user can also click on the background of the visualization in order to pan through the visualization and inspect other states and transitions. The visualization is also interactive. The user can grab a state and move it around to a desired position.

We had some performance issues that were associated with the rendering very large state spaces. The force layout keeps adjusting the graph until it reaches a fixpoint. The problem was that as the state space grew, there were more and more objects that had to be rendered. The force calculates the position of the vertices in the graph iteratively. By default, the graph is rerendered in every iteration of the calculation of the graph layout. When the number of states in a state space is very large, the rendering and therefore the calculation of the layout, takes a lot of resources. This affected not only the appearance of the visualization, but it also cost enough resources that the whole eclipse plugin became unresponsive. The solution we found to the problem was to allow the user to turn off the rendering of the graph. The user can do this by selecting the play/pause button in the upper left hand corner of the visualization (see Figure \ref{userSelect}). The visualization will then freeze in the position that is is currently drawn. The force layout, however, will continue running in the background, so if the user decides to turn on the rendering, the graph will hopefully have had time to stabilize.

\begin{center}
\begin{figure}[h!]
\centering
\includegraphics[width=10cm]{bilder/onestate.png}
\caption{By zooming in, the user can inspect individual nodes.}
\label{onestate}
\end{figure}
\end{center}

When new states are added to the state space, these are added to the graph when the visualization receives them in the next poll. A node is inserted at the same position as the preceding state in the graph. The graph is then updated. This results in a nice animation because the new states will appear to pop outward from their parent state.

Status about the invariant is available in the graph based on the color of the nodes. If an invariant violation is present, the state is colored red. If the invariant is ok, the state is colored green. Otherwise, if the invariant has not yet been calculated for the given state, the node is colored gray. The labels for the states are the values of the variables for the given state.

The state spaces often grow exponentially. This is known as the state space explosion problem. For this reason, a visualization of the entire state space can be too much information for a user to process at any given time. In order to help understand the overall behavior of the state space, we have also made smaller graphs available that are derived from the original state space and can be more useful for the user. The user can choose from these visualizations in the dropdown menu in the upper left hand corner (see Figure \ref{userSelect}).

\begin{center}
\begin{figure}[h!]
\centering
\includegraphics[width=8cm]{bilder/menu.png}
\caption{The user can select the desired visualization and play and pause the rendering.}
\label{userSelect}
\end{figure}
\end{center}

\subsubsection{Signature Merged State Spaces}

The signature merge algorithm is the first of two algorithms for reducing the state space that have been implemented in this work (see Figure \ref{sigmerge}). 

\begin{center}
\begin{figure}[h!]
\centering
\includegraphics[width=11cm]{bilder/sigmerge.png}
\caption{D3 Visualization of signature merge for the Scheduler example}
\label{sigmerge}
\end{figure}
\end{center}


The algorithm works by merging all of the states which have the same outgoing transitions. This creates a state space that is considerably smaller but that still preserves information about the operations that are enabled for a given state. 

If the user wants to use the GraphViz algorithms and rendering engine, there is also support for visualizing the DOT representation of the signature merged state space that is generated from ProB (see Figure \ref{sigmergeDotty}). 

\begin{center}
\begin{figure}[h!]
\centering
\includegraphics[width=14cm]{bilder/dotty-sigmerge.png}
\caption{GraphViz powered visualization of the signature merge algorithm for the Scheduler example}
\label{sigmergeDotty}
\end{figure}
\end{center}

Rendering of the DOT graph specification is done using the Viz.js JavaScript library. The visualization supports zooming and panning in the same way as the D3 powered visualizations do.

The graph that is generated with the signature merge algorithm differs based on the transitions that are of the interest of to the user. By default, the algorithm is calculated using all of the transitions that are present in the model. However, it also possible to select the desired transitions for the calculation of the algorithm. In order to allow the user to select the transitions of interest, we have implemented a small user interface that pops up when the user clicks on the settings icon (see Figure \ref{sigMergeUI}). The settings icon only becomes available when the user is in a mode dealing with the signature merge algorithm. The label for the nodes in this graph consists of a list of the outgoing transitions from the node and the number of states that have been merged into the node.  

\begin{center}
\begin{figure}[h!]
\centering
\includegraphics[width=8cm]{bilder/sigMergeUI.png}
\caption{User interface to chose events for signature merge.}
\label{sigMergeUI}
\end{figure}
\end{center}


\subsubsection{Transition Diagrams}


The other reduction algorithm that is supported in this implementation is the creation of transition diagrams. In order to perform this algorithm, ProB receives an expression from the user. Then ProB calculates all of the possible solutions to the expression in the scope of the model that is being animated. These become the vertices in the graph. The edges in the graph show the transitions that change the value of the given expression to that of the value shown in the target state (see Figure \ref{transdiag}).

If the user prefers the GraphViz represenation over the D3 powered visualization, it is also possible to generate a GraphViz based visualization for the transition diagram (see Figure \ref{transdiagDotty}). As with the signature merged state space, the UI for the GraphViz powered transition diagram is the same as that for the D3 powered visualization, and the visualization supports zooming and panning. Although the GraphViz algorithms are inherently offline, these visualizations make use of the visualization framework and are recalculated and rerendered every time that a change takes place within the state space.

When the user chooses to create a transition diagram from the menu, a prompt appears. This is how the user can specify the initial expression for the calculation of the transition diagram. Once the graph is calculated and rendered, a text field appears next to the drop down menu (see Figure \ref{transDiagUI}). If the user wants to change the expression that is being visualized, they can input a new expression here and submit it. The algorithm will then be recalculated and the graph will be rerendered.

\begin{center}
\begin{figure}[h!]
\centering
\includegraphics[width=8cm]{bilder/transDiag-UI.png}
\caption{The user can input a new expression to recalculate the transition diagram.}
\label{transDiagUI}
\end{figure}
\end{center}


\begin{center}
\begin{figure}[h!]
\centering
\includegraphics[width=14cm]{bilder/transdiag.png}
\caption{D3 Visualization of the transition diagram of \texttt{active} in the Scheduler example}
\label{transdiag}
\end{figure}
\end{center} 

\pagebreak

\begin{center}
\begin{figure}[h!]
\centering
\includegraphics[width=14cm]{bilder/transdiag-dotty-wo.png}
\caption{GraphViz powered visualization of the transition diagram of \texttt{active} in the Scheduler model}
\label{transdiagDotty}
\end{figure}
\end{center}

\subsection{Visualization of a B-type Formula}

ProB already supported the functionality of expanding a formula into its subformulas and finding its value at a given state. However, for any given formula, only the subformulas directly under the desired formula would be calculated. This algorithm was adapted so that all the subformulas are calculated recursively and a tree structure is built automatically. This structure is then cached, and can then be evaluated for any given state. The user can create a new visualization for a formula by selecting \textsf{Analyze > Open Visualization of Formula} from the \textsf{ProB} menu and entering the desired formula in the prompt that pops up, or by right clicking on a formula in the \textsf{State Inspector} view and selecting \textsf{Open Visualization of Formula}.

The final visualization is interactive (see Figure \ref{predicate}). If a formula has subformulas, the user can select it from within the visualization to expand or to retract the subformulas. The subformulas are always either predicates or expressions. If they are expressions, they are colored white or light grey depending on whether they have subformulas or not. If the formula is a predicate that has evaluated to true for the given formula, the node is colored green. If the formula is a predicated that has evaluated to false, node is colored in red. The value of the given formula is also printed beneath the formula. This allows the user to visually identify the parts of the formulas and their given values. 

\begin{center}
\begin{figure}[h!]
\centering
\includegraphics[width=14cm]{bilder/invariant.png}
\caption{Visualization of the invariant of the Scheduler model}
\label{predicate}
\end{figure}
\end{center}

In this visualization, the D3 tree layout is used. The expanding and collapsing of the nodes takes place with a simple JavaScript function. This design of this visualization is based on the Collapsible Tree Layout from the D3 website (see Appendix \ref{appendix:tree}). By harnessing the power of the D3 zoom behavior, it is also possible to zoom in and out of the visualization and to pan the image to inspect it closer. When the current state in the animation changes, the formula is recalculated and the visualization is redrawn.

\subsection{Visualization of the value of a formula over time}

The last visualization that we have created in the course of this work is the visualization of the value of a given formula over the course of an animation.
This can be especially useful if the user is interested in analyzing models with a temporal element. 

To create this visualization, the user can select \textsf{Analyze > Open Time vs Value Visualization} from the \textsf{ProB} menu. A prompt will then pop up and the user can specify the formula that is to be evaluated for all the states in the current animation. Optionally, the user can also specify a second expression which will represent the temporal element for the model. In this case, the values of the first formula will be plotted against the values of the expression that the user has inputted. Otherwise, the values of the first formula will be plotted against the number of animation steps.

The value pairs that are produced through this evaluation are processed by D3 to produce a simple line plot (see Figure \ref{timeVsValue}). If the current state changes, the formula is recalculated and a new plot is produced.

\begin{center}
\begin{figure}[h!]
\centering
\includegraphics[width=15cm]{bilder/vOtOver.png}
\caption{Visualization of the value of \texttt{level} and \texttt{pump} over \texttt{time} for the water tank example.}
\label{timeVsValue}
\end{figure}
\end{center}

It is possible to visualize multiple formulas at the same time. In order to add a formula to the visualization, the user can input a new formula in the text box in the upper left hand corner (see Figure \ref{timeVsValueUI}).

\begin{center}
\begin{figure}[h!]
\centering
\includegraphics[width=8cm]{bilder/timeVValueUI.png}
\caption{The user can add an expression to the visualization.}
\label{timeVsValueUI}
\end{figure}
\end{center}

There are two modes for viewing the line plots. The first mode plots all of values in the same line plot. If the user want to view each individual line separately, he can select the \textsf{Toggle Mode} button in the lower right corner of the visualization. This will produce a separate line plot for each formula (see Figure \ref{vOtEach}).

It is possible to visualize predicates and expressions that take on integer or boolean values. If the formula that is being visualized is a predicate or an expression that takes on a boolean value, the resulting value is mapped to an integer value. If all of the formulas are being drawn in the same line plot, the boolean value TRUE will be mapped to the maximum value of all the data sets. Otherwise, it will be mapped to 1. The boolean value FALSE is always mapped to 0.

\begin{center}
\begin{figure}[h!]
\centering
\includegraphics[width=15cm]{bilder/vOtEach.png}
\caption{Visualization of separate line charts for each inputted formula in the water tank example.}
\label{vOtEach}
\end{figure}
\end{center}