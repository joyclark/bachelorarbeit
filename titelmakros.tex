

%%%%%%%%%%%%%%%%%%%%%%%%%%%%%%%%%%%%%%%%%%%%%%%%%%%%%%%%%%%
% Obere Titelmakros. Editieren Sie diese Datei nur, wenn
% Sie sich ABSOLUT sicher sind, was Sie da tun!!!
% (Z.B. zum Abaendern der BA-Vorlage in eine MA-Vorlage)
% Uni Duesseldorf
% Lehrstuhl fuer Datenbanken und Informationssysteme
% Version 2.2 - 2.3.2010
%%%%%%%%%%%%%%%%%%%%%%%%%%%%%%%%%%%%%%%%%%%%%%%%%%%%%%%%%%%
\newcommand{\AN}{twoside}
\newcommand{\AUS}{}
%\newcommand{\englisch}{}
%\newcommand{\deutsch}{\usepackage[german]{babel}}

%% Die folgenden auskommentierten Optionen dienen der automatischen
%% Erkennung des Latex-Kompilers und dem Setzen der davon abhängigen
%% Einstellungen. Bei Problem z.B. mit dem Einbinden von verschiedenen
%% Grafiktypen bei Verwendung von PdfLatex oder Latex, einfach die
%% verschiedenen \usepackage(s) ausprobieren. (Mit diesen Einstellungen
%% funktionierte diese Vorlage bei der Verwenundg von latex.exe als
%% Kompiler bei den meisten Studierenden.)

%\newif\ifpdf \ifx\pdfoutput\undefined
%\pdffalse % we are not running pdflatex
%\else
%\pdfoutput=1 % we are running pdflatex
%\pdfcompresslevel=9 % compression level for text and image;
%\pdftrue \fi

\documentclass[11pt,a4paper, \zweiseitig]{article}



%\usepackage[iso]{umlaute}
\usepackage[utf8x]{inputenc}
\usepackage{palatino} % palatino Schriftart
%\usepackage{makeidx} % um ein Index zu erstellen
\usepackage{tocbibind}
\usepackage[T1]{fontenc} %fuer richtige Trennung bei Umlauten
\usepackage{fancybox} % fuer die Rahmen
\usepackage{shortvrb}
\usepackage{ifthen}
\ifthenelse{\equal{\sprache}{deutsch}}{\usepackage[ngerman]{babel}}{}

\usepackage{a4wide} % ganze A4 Weite verwenden
\usepackage{amsthm}
%\ifpdf
%\usepackage[pdftex,xdvi]{graphicx}
%\usepackage{thumbpdf} %thumbs fuer Pdf
%\usepackage[pdfstartview=FitV]{hyperref} %anklickbares Inhaltsverzeichnis
%\else
%\usepackage[dvips,xdvi]{graphicx}
\usepackage{graphicx}
\usepackage{hyperref} %anklickbares Inhaltsverzeichnis
%\fi

\usepackage{pdfpages}

%%%%%%%% CODE LISTINGS %%%%%%%%
\usepackage{listings}
\usepackage{courier}
\lstset{
         basicstyle=\footnotesize\ttfamily, % Standardschrift
         %numbers=left,               % Ort der Zeilennummern
         numberstyle=\tiny,          % Stil der Zeilennummern
         %stepnumber=2,               % Abstand zwischen den Zeilennummern
         numbersep=5pt,              % Abstand der Nummern zum Text
         tabsize=2,                  % Groesse von Tabs
         extendedchars=true,         %
         breaklines=true,            % Zeilen werden Umgebrochen
         keywordstyle=\bfseries,
         frame=b,         
         stringstyle=\ttfamily, % Farbe der String
         showspaces=false,           % Leerzeichen anzeigen ?
         showtabs=false,             % Tabs anzeigen ?
         xleftmargin=17pt,
         framexleftmargin=17pt,
         framexrightmargin=5pt,
         framexbottommargin=4pt,
         showstringspaces=false      % Leerzeichen in Strings anzeigen ?        
 }
 \lstloadlanguages{
         Java
 }
 \usepackage{caption}
\DeclareCaptionFont{white}{\color{white}}
\DeclareCaptionFormat{listing}{\colorbox[cmyk]{0.43, 0.35, 0.35,0.01}{\parbox{\textwidth}{\hspace{15pt}#1#2#3}}}
\captionsetup[lstlisting]{format=listing,labelfont=white,textfont=white, singlelinecheck=false, margin=0pt, font={bf,footnotesize}}

%%%%%%%%%%%%%%%%%%%%%%% Massangaben fuer die Arbeit %%%%%%%%%%%%%%%
\setlength{\textwidth}{15cm}

\setlength{\oddsidemargin}{35mm}
\setlength{\evensidemargin}{25mm}

\addtolength{\oddsidemargin}{-1in}
\addtolength{\evensidemargin}{-1in}

%\makeindex

\begin{document}

%\setcounter{secnumdepth}{3} %Nummerieren bis in die 3. Ebene
%\setcounter{tocdepth}{3} %Inhaltsverzeichnis bis zur 3. Ebene

\pagestyle{headings}

\sloppy % LaTeX ist dann nicht so streng mit der Silbentrennung
%~ \MakeShortVerb{\§}

\parindent0mm
\parskip0.5em


{
\textwidth170mm 
\oddsidemargin30mm 
\evensidemargin30mm 
\addtolength{\oddsidemargin}{-1in}
\addtolength{\evensidemargin}{-1in}

\parskip0pt plus2pt

% Die Raender muessen eventuell fuer jeden Drucker individuell eingestellt
% werden. Dazu sind die Werte fuer die Abstaende `\oben' und `\links' zu
% aendern, die von mir auf jeweils 0mm eingestellt wurden.

%\newlength{\links} \setlength{\links}{10mm}  % hier abzuaendern
%\addtolength{\oddsidemargin}{\links}
%\addtolength{\evensidemargin}{\links}

\begin{titlepage}
\vspace*{-1.5cm}
  \raisebox{17mm}{
    \begin{minipage}[t]{70mm}
      \begin{center}
        %\selectlanguage{german}
        {\Large INSTITUT FÜR INFORMATIK\\}
        {\normalsize
          Softwaretechnik und Programmiersprachen\\
        }
        \vspace{3mm}
        {\small Universitätsstr. 1 \hspace{5ex} D--40225 Düsseldorf\\}
     \end{center}
    \end{minipage}
  }
  \hfill
  \includegraphics[width=130pt]{bilder/HHU_Logo}
  \vspace{14em}

% Titel
  \begin{center}
      	\baselineskip=55pt
    	\textbf{\huge \titel}
  	 	\baselineskip=0 pt
   \end{center}

  %\vspace{7em}

\vfill

% Autor
  \begin{center}
    \textbf{\Large
      \bearbeiter
    }
  \end{center}

  \vspace{35mm}
 
% Prüfungsordnungs-Angaben
  \begin{center}
    %\selectlanguage{german}
    
%%%%%%%%%%%%%%%%%%%%%%%%%%%%%%%%%%%%%%%%%%%%%%%%%%%%%%%%%%%%%%%%%%%%%%%%%
% Ja, richtig, hier kann die BA-Vorlage zur MA-Vorlage gemacht werden...
% (nicht mehr nötig!)
%%%%%%%%%%%%%%%%%%%%%%%%%%%%%%%%%%%%%%%%%%%%%%%%%%%%%%%%%%%%%%%%%%%%%%%%%
    {\Large \arbeit}

    \vspace{2em}

    \begin{tabular}[t]{ll}
      Beginn der Arbeit:& \beginndatum \\
      Abgabe der Arbeit:& \abgabedatum \\
      Gutachter:         & \erstgutachter \\
                         & \zweitgutachter \\
    \end{tabular}
  \end{center}

\end{titlepage}

}

%%%%%%%%%%%%%%%%%%%%%%%%%%%%%%%%%%%%%%%%%%%%%%%%%%%%%%%%%%%%%%%%%%%%%
\clearpage
\begin{titlepage}
  ~                % eine leere Seite hinter dem Deckblatt
\end{titlepage}
%%%%%%%%%%%%%%%%%%%%%%%%%%%%%%%%%%%%%%%%%%%%%%%%%%%%%%%%%%%%%%%%%%%%%
\clearpage
\begin{titlepage}
\vspace*{\fill}

\section*{Erklärung}

%%%%%%%%%%%%%%%%%%%%%%%%%%%%%%%%%%%%%%%%%%%%%%%%%%%%%%%%%%%
% Und hier ebenfalls ggf. BA durch MA ersetzen...
% (Auch nicht mehr nötig!)
%%%%%%%%%%%%%%%%%%%%%%%%%%%%%%%%%%%%%%%%%%%%%%%%%%%%%%%%%%%

Hiermit versichere ich, dass ich diese \arbeit~
selbstständig verfasst habe. Ich habe dazu keine anderen als die
angegebenen Quellen und Hilfsmittel verwendet.

\vspace{25 mm}

\begin{tabular}{lc}
Düsseldorf, den \abgabedatum \hspace*{2cm} & \underline{\hspace{6cm}}\\
& \bearbeiter
\end{tabular}

\vspace*{\fill}
\end{titlepage}

%%%%%%%%%%%%%%%%%%%%%%%%%%%%%%%%%%%%%%%%%%%%%%%%%%%%%%%%%%%%%%%%%%%%%
% Leerseite bei zweiseitigem Druck
%%%%%%%%%%%%%%%%%%%%%%%%%%%%%%%%%%%%%%%%%%%%%%%%%%%%%%%%%%%%%%%%%%%%%

\ifthenelse{\equal{\zweiseitig}{twoside}}{\clearpage\begin{titlepage}
~\end{titlepage}}{}

%%%%%%%%%%%%%%%%%%%%%%%%%%%%%%%%%%%%%%%%%%%%%%%%%%%%%%%%%%%%%%%%%%%%%
\clearpage
\begin{titlepage}

%%% Die folgende Zeile nicht ändern!
\section*{\ifthenelse{\equal{\sprache}{deutsch}}{Zusammenfassung}{Abstract}}
%%% Zusammenfassung:

The ProB tool performs the animation and verification of models written in several different formal specification languages. Here we propose a method for the generation of dynamic visualizations to help users better understand the results that are produced from ProB. In order to do this, we used the D3 JavaScript library to generate visualizations based on the data produced by ProB. In this work, we will present three new visualizations that are available for the user. We will also describe the new visualization framework that has been implemented to allow the user to interact with the visualizations and to allow the visualizations to be updated when changes in ProB take place. In the future, it should be relatively easy for developers to extend the existing framework to make new visualizations.


%%%%%%%%%%%%%%%%%%%%%%%%%%%%%%%%%%%%%%%%%%%%%%%%
% Untere Titelmakros. Editieren Sie diese Datei nur, wenn Sie sich
% ABSOLUT sicher sind, was Sie da tun!!!
%%%%%%%%%%%%%%%%%%%%%%%%%%%%%%%%%%%%%%%%%%%%%%%
\vspace*{\fill}
\end{titlepage}

%%%%%%%%%%%%%%%%%%%%%%%%%%%%%%%%%%%%%%%%%%%%%%%%%%%%%%%%%%%%%%%%%%%%%
% Leerseite bei zweiseitigem Druck
%%%%%%%%%%%%%%%%%%%%%%%%%%%%%%%%%%%%%%%%%%%%%%%%%%%%%%%%%%%%%%%%%%%%%
\ifthenelse{\equal{\zweiseitig}{twoside}}
  {\clearpage\begin{titlepage}~\end{titlepage}}{}
%%%%%%%%%%%%%%%%%%%%%%%%%%%%%%%%%%%%%%%%%%%%%%%%%%%%%%%%%%%%%%%%%%%%%
\clearpage \setcounter{page}{1}\pagenumbering{roman}
\tableofcontents
%\enlargethispage{\baselineskip}
\clearpage
%%%%%%%%%%%%%%%%%%%%%%%%%%%%%%%%%%%%%%%%%%%%%%%%%%%%%%%%%%%%%%%%%%%%%
% Leere Seite, falls Inhaltsverzeichnis mit ungerader Seitenzahl und 
% doppelseitiger Druck
%%%%%%%%%%%%%%%%%%%%%%%%%%%%%%%%%%%%%%%%%%%%%%%%%%%%%%%%%%%%%%%%%%%%%
\ifthenelse{ \( \equal{\zweiseitig}{twoside} \and \not \isodd{\value{page}} \)}
	{\pagebreak \thispagestyle{empty} \cleardoublepage}{\clearpage}
\clearpage \setcounter{page}{1}\pagenumbering{arabic}



